
\documentclass{book}
\usepackage{amsmath, amsthm, graphicx}
\usepackage[english]{babel}

\title{Mechanics of Machines for Automation}
\author{dante piotto}
\date{spring semester 2023}

\begin{document}

\maketitle

\chapter{introduction}

\section{systems and models}


\section{dofs and constraints}

kutzbach-grubler formula (spatial motion):
\begin{itemize}
    \item $m$: number of mobile bodies excluding the fixed frame
    \item $n$: number of joints
    \item $v_k$ degrees of constraint provided by the k-th joints
    \item $v$ overall degree of constraint of the system
\end{itemize}
\[
    n=6m-v \quad \text{with}\quad v=\sum_{k=1}^{n}v_k
\]
\chapter{mathematical tools}

\section{vectors and matrices}

\section{differentiation}


\section{difference beteween holonomic and non holonomic constraints}
holonomic: starting from position A and arriving to position B the configurations in A and B are the same whatever the path taken to go from A to B\\
with non holonomic constraints thA end configuration depends on the path taken and cannot be determined by the positions in the starting and end point alone.\\
Example: rotate a wheel with no slipping by 90 degrees about the y axis and then by 90 degreees about the z axis, the end position does not depend on the order of the rotations however the configuration does. 


\section{Four bar linkage}
ratio between the velocities of the driving link and the driven links: trasmission ratio. Can be fixed in time or not. Also called first-order kinematic coefficients.
\\Five bar linkage\\
calculations needed for exam: slide 36 no need for expression of D,E,F coeffs.
fwd kinematics:subtracting equations to get linear relation between x and y, following quadratic equation and considerations, up-down configurations








\section{Delta bot}
schematics for a single arm + figure on the right in slide 8\\
Closure equations and mathematical developments up to slide 14, type of manipulations needed and what type of relation is obtained from slide 15 to 16, how eq obtained in slide 16 is solved and the 2 types of solution with drawing, pretty much all inverse kinematics with drawings of 2 solutions.
\section{Elastic Forces}
Equivalent spring constants:
rotational springs:
\[
    K_{eq}=\frac{M}{\theta}=\frac{d^2U}{d\theta^2}
\]
linear springs:
\[
    K_{eq}=\frac{F}{x}=\frac{d^2U}{dx^2}
\]



\end{document}
