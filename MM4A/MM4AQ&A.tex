\documentclass{article}
\usepackage{amsmath, amsthm, graphicx}
\usepackage[english]{babel}


\title{Mechanics of Machines for Automation\\Exam Questoins and Answers}
\author{Dante Piotto}
\date{spring semester 2023}

\begin{document}

\maketitle

\section{Rolling cylinder: schematic; definition of slipping and not-slipping conditions; degrees of freedom
and constraint equations in the cases of slipping and not-slipping}

\section{3-wheel (tricycle) model with no slipping: schematic; degrees of freedom; constraint equation in
velocity form; velocity field; equations describing the evolution of system configuration}

\section{Finite motion in 3D space: coordinate transformations; definition and properties of rotation
matrices; elementary rotation matrices; non-commutativity of finite rotations; composition of
rotation matrices. Describe with schematics and equations.}

\section{Finite motion in 3D space: coordinate transformations; definition of homogeneous transformation
matrices; inverse of homogeneous transformation matrices; elementary homogeneous
transformation matrices; composition of homogeneous transformation matrices. Describe with
schematics and equations.}

\section{Infinitesimal motion in 3D space: rate of change of a vector and angular velocity; additive
properties of angular velocity. Describe with schematics and equations}

\section{Infinitesimal motion in 3D space: vector derivatives with respect to different reference frames;
angular acceleration; additive properties of angular acceleration. Describe with schematics and
equations.}

\section{Infinitesimal motion in 3D space: vector second derivatives in different frames; rigid body
velocities; rigid body acceleration. Describe with schematics and equations.}

\section{Four-bar linkage: schematic; degrees of freedom; equations for position, velocity and acceleration
analyses}

\section{Symmetric five-bar linkage: schematic; degrees of freedom; equations for inverse kinematic
analysis; equations for forward kinemetric analysis; equations for velocity and acceleration analyses}

\section{Symmetric five-bar linkage: schematic; degrees of freedom; workspace analysis with equations and
graphical representation; singularity analysis with equations and graphical representation}

\section{Derive the equation expressing the equivalent spring constants for springs in series and in parallel;
derive the equations describing the deflection of slender beams}

\section{Rigid body dynamics in 3D space. Derive the equation for: center of mass; linear momentum;
angular momentum with respect to points that belong to the body, either fixed or moving with
respect to an inertial frame.}

\section{Rigid body dynamics in 3D space. Derive the equation for: resultant of inertia forces; resultant of
moment of inertia forces with respect to points that belong to the body, either fixed or moving
with respect to an inertial frame}

\section{Rigid body dynamics in 3D space: definition and properties of the inertia matrix; derive the
equation for the kinetic energy of a body that is moving with respect to an inertial frame; kinetic
energy and work of inertia forces}

\section{Derive the Lagrange equations for holonomic system with n degrees of freedom}

\section{Derive the Lagrange equations for constrained system with m generalized coordinates subjected to
v constraints}

\section{(maybe not) Dynamic equivalent systems of masses for bodies moving in space and in space and in plane
(including an example). Provide a schematic, derive the equations and describe their use}

\section{Application of the Lagrange approach to determine the motion equations of a lumped parameter
model of a 1 d.o.f transmission: without dissipations and with dissipation (both direct and inverse
motion)}

\section{(maybe not) Motor-load coupling: motor mechanical characteristic; motor operation ranges; load operation
ranges. Describe with schematics.}

\section{Motor-load coupling for systems operating at quasi constant speed: stability at steady state; start
and stop transients (with the determination of starting time). Describe with schematics and
equations.}

\section{Schematics, equations and plot of the responses of a vehicle powered by a permanent magnet DC
motor and going up-hill for the cases where: vehicle and motor dynamics can be neglected; only
motor dynamics can be neglected; none of vehicle and motor dynamics can be neglected}

\section{Schematics and equations describing the effect of speed reducer in motor-load coupling for
systems operating in steady state for the following cases: motor as ideal speed generator; motor as
ideal torque generator; motor as ideal power generator; general real motor}

\section{Optimal transmission ratio during transients: determination of the transmission ratio that
minimizes motor torque requirement; effect of non-optimal transmission ratio on motor torque
and on motor power. Describe with schematics and equations.}

\section{Schematics, equations and plot of the responses of a system operating with periodic regime for the
following cases: inertia is negligible; inertia is not negligible and motor stiffness is negligible; both
inertia and motor stiffness are not negligible}

\section{Use of flywheels to reduce motor size and speed oscillation in systems operating with periodic
regime. Describe with schematics and equations.}

\section{Delta robot kinematics: generalities (diagrams, DOF), closure equations, position analysis (direct
and inverse kinematics).}

\section{Schematics, equations and plots of the free response of first order (mass-damper or springdamper) systems with 1 d.o.f.}

\section{ Schematics, equations and plots of the free response of second order (mass-spring-damper)
systems with 1 d.o.f. (for all cases of overdamping, critically damping and underdamping)}

\section{Logarithmic decrement method to estimate the damping factor in second-order (mass-springdamper) systems with 1 d.o.f. Describe with schematics and equations.}

\section{Free vibration of a U-tube manometer via the Rayleigh’s energy method: schematic and equations
for tubes with constant and variable cross-section}

\section{Schematic, equations and plots of the response of a second-order (mass-spring-damper) system
with 1 d.o.f. subjected to a step force}

\section{Schematic, equations and plots of the response of a second-order (mass-spring-damper) system
with 1 d.o.f. subjected to an impulse force}

\section{Schematic, equations and plots of the response of a second-order (mass-spring-damper) system
with 1 d.o.f. subjected to a general force excitation via the convolution integral approach}

\section{Schematic, equations and plots of the response of a second-order (mass-spring-damper) system
with 1 d.o.f. subjected to Coulomb friction}

\section{Schematic, equations and plots of the response of a second-order (mass-spring-damper) system
with 1 d.o.f. subjected to harmonic excitation}

\section{Damping estimation by the bandwidth method (half-power method) in second-order (mass-springdamper) systems with 1 d.o.f.. Describe with schematics and equations, including also the
calculation of the energy dissipated in viscous damping.}

\section{Time response to harmonic excitation at system natural frequency in undamped second-order
(mass-spring-damper) systems with 1 d.o.f.. Describe with schematics and equations.}

\section{Schematic, equations and plots of the response of a second-order (mass-spring-damper) system
with 1 d.o.f. to a harmonic excitation due to imbalances}

\section{Schematic, equations and plots of the vibration response of the Jeffcott rotor (whirling of rotating
shafts)}

\section{Transmitted force in second-order (mass-spring-damper) system with 1 d.o.f. subjected to
harmonic excitation: schematic, equations and plots}

\section{Schematic, equations and plots of the response of a second-order (mass-spring-damper) system
with 1 d.o.f. to the harmonic motion of the base (base excitation)}

\section{Energy extraction from an oscillating second-order (mass-spring-damper) system subjected to
harmonic excitation: schematic, equations and plots}

\section{Effect of spring mass on the natural frequency of second-order (mass-spring-damper) systems:
schematic and equations (via the Rayleigh’s energy method)}

\section{Free response of an undamped multi-d.o.f. mass-spring system: schematic, equations, natural
frequencies, vibration modes, rigid body modes}

\section{Decoupling of the equations of motion for undamped multi-d.o.f. mass-spring systems:
orthogonality of the modal vectors, modal equations of motion and modal response, mode
participation and isolation}

\section{ Free motion response of damped multi d.o.f. mass-damper-spring systems: derive the equations of
the classical solution for both overdamped and underdamped cases}

\section{Forced motion response of damped multi d.o.f. mass-damper-spring systems: derive the equations
of the state form formulation and the equations of the approximate solution}

\section{Forced motion response of damped multi d.o.f. mass-damper-spring systems: derive the equations
for the response to harmonic excitation with the classical approach}

\section{Motion response of the 2-d.o.f. coupled pendulum: schematic; derive the equations of motion;
natural frequencies and mode shapes; discuss the beat phenomenon with equations}

\section{Derive the motion equations for the response of a 6-d.o.f. mass-damper-spring system with
proportional damping to harmonic forcing}

\section{The Rayleigh’s method to find the approximate value of the fundamental natural frequency of an
undamped mass-spring n-d.o.f. system. Derive the equations and describe their use.}












\end{document}